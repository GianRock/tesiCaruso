% !TEX encoding = UTF-8
% !TEX TS-program = pdflatex
% !TEX root = ../tesi.tex
% !TEX spellcheck = it-IT

%*******************************************************
% Introduzione
%*******************************************************
\cleardoublepage
\chapter*{Abstract}
Twitter è ad oggi il servizio di micro-blogging più utilizzato in assoluto. Con circa 284 milioni di utenti attivi al mese, ogni giorno vengono prodotti oltre 500 milioni di tweets. Gli utenti di Twitter, possono pubblicare degli status, o \emph{tweets}, non  più lunghi di 140 caratteri. Se questo vincolo da una parte costituisce un forte limite, dall'altro rappresenta una delle caratteristiche fondamentali di Twitter: \emph{l'immediatezza}. Questo limite costringe gli utenti ad produrre messaggi molto sintetizzati, quasi come slogan, che quindi sono più facili da diffondere. Grazie a queste caratteristiche, qualsiasi persona che assiste o è coinvolta in un \emph{evento}, è in grado di diffondere informazioni in \emph{real-time}.
Talvolta, i tweets (Twitter microblog posts) diffondono notizie anche più velocemente dei media tradizionali (come la morte di Micheal Jackson \footnote{ttp://www.dailymail.co.uk/sciencetech/article-1195651/How-Michael-Jacksons-death-shut-
Twitter-overwhelmed-Google–killed-Jeff-Goldblum.html}).
Bisogna però sottolineare che questi tweet che possono riflettere eventi, rappresentano solo una piccola percentuale di tutti i tweet prodotti. La maggior parte infatti, è costituita da status personali, messaggi anche privi di senso, spam. Risulta quindi necessario un sistema che sia capace di scoprire \lq\lq eventi\rq\rq o \lq\lq topics\rq\rq   da questo flusso di dati.  Affinché questi eventi abbiano una qualche utilità, è altresì necessario, che siano  rilevati con una bassa latenza. I classici algoritmi di Data Mining, non riescono a scalare all'aumentare della mole dei dati . Questo lavoro di tesi ha l'obiettivo di applicare algoritmi di data Mining per la scoperta di eventi, attraverso un'architettura di calcolo distribuita: Apache SPARK. 

