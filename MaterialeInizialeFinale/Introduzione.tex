% !TEX encoding = UTF-8
% !TEX TS-program = pdflatex
% !TEX root = ../tesi.tex
% !TEX spellcheck = it-IT

%*******************************************************
% Introduzione
%*******************************************************
\cleardoublepage
\chapter*{Introduzione}
Twitter è ad oggi il servizio di micro-blogging più utilizzato in assoluto, con circa 284 milioni di utenti attivi al mese, ogni giorno vengono prodotti oltre 500 milioni di tweets. Gli utenti di Twitter, possono pubblicare degli status, o \emph{tweets}, non  più lunghi di 140 caratteri. Se questo vincolo da una parte costituisce un forte limite, dall'altro rappresenta una delle caratteristiche fondamentali di Twitter: \emph{l'immediatezza}. Questo limite costringe gli utenti ad produrre messaggi molto sintetizzati, quasi come slogan, che quindi sono più facili da diffondere. Grazie a queste caratteristiche, qualsiasi persona che assiste o è coinvolta in un \emph{evento}, è in grado di diffondere informazioni in \emph{real-time}.
Talvolta, i tweets (Twitter microblog posts) diffondono notizie anche più velocemente dei media tradizionali (come la morte di Micheal Jackson \footnote{ttp://www.dailymail.co.uk/sciencetech/article-1195651/How-Michael-Jacksons-death-shut-
Twitter-overwhelmed-Google–killed-Jeff-Goldblum.html}).
Bisogna però sottolineare che questi tweet che possono riflettere eventi, rappresentano solo una piccola percentuale di tutti i tweet prodotti. La maggior parte infatti, è costituita da status personali, messaggi anche privi di senso, spam. Risulta quindi necessario un sistema che sia capace di scoprire \lq\lq eventi\rq\rq o \lq\lq topics\rq\rq   da questo flusso di dati. Scoprire nuovi eventi o topics da Twitter, non è affatto un task banale sia 
sia la mole dei dati (oltre 400 milioni di tweet giornalmente), che per la natura stessa dei tweets. Se da un lato il limite a 140 caratteri ne rende più semplice la diffusione on-line, dall'altro complica ulteriormente il task, poiché gli utenti spesso,proprio a causa di tale limite,ricorrono a slang, vocaboli OOV\footnote{Out Of Vocabolary} o emoticons.
\section*{Twitter}
Twitter è ad oggi il servizio di "microblogging" più diffuso e con il più alto tasso di crescita.  Negli anni la sua popolarità crescente ha anche attirato anche un alto numero di ricercatori, come si può notare dall'alto numero di articoli riguardanti Twitter che sono stati pubblicati in numerosi campi di ricerca.
Sebbene il termine  “microblog” spesso possa indurre intendere tale servizio come una versione "micro" di un blog, sono due due media molto diversi fra loro \cite{export:69500}. I blog infatti sono progettati, principalmente per permettere ad utenti, di fornire commenti e opinioni su topic di cui sono esperti, gli autori devono anche garantire una certa validità dei contenuti. Dall'altra parte, i microblog, come Twitter, sono invece pensati per permettere di condividere opinioni, news, ma in maniera molto concisa (max 140 caratteri) proprio per far sì che si abbia una  diffusione tempestiva delle informazioni. Proprio grazie a questa caratteristica, i tweets possono essere pubblicati mediante dispositivi mobili, consentendo a chiunque sia testimone di un qualsiasi evento di diffondere la notizia in real-time. Twitter inoltre è anche un servizio di social networking, ogni utente può ricevere gli aggiornamenti ("follow")  di altri utenti senza previa approvazione. Questa relazione è asimmetrica e può essere concettualizzata come una Directed social network o \emph{follower network} 
\subsection*{Twitter come fonte di informazione}
Molte notizie sono state diffuse su Twitter anche prima della diffusione sui media classici. Uno degli esempi più significati è stato rappresentato dalla notizia della di Michael Jackson del 2009. Alle 2:26pm
del 24 Giungo 2009, la notizia trapelò su Twitter e fu diffusa in una maniera così virale che che Google la identificò come un attacco hacker.t . La validità della notizia fù verificata da Google solo 25 minuti dopo,   solo allora i media mainstream iniziarono a far diffondere la notizia \footnote{ttp://www.dailymail.co.uk/sciencetech/article-1195651/How-Michael-Jacksons-death-shut-
Twitter-overwhelmed-Google–killed-Jeff-Goldblum.html}.Anche nel caso del terremoto in Abruzzo del 6 aprile 2009, gli utenti Twitter hanno segnalato la notizia prima dei media tradizionali. 
