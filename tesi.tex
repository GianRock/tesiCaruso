\documentclass[a4paper,12pt,oneside]{book}
\usepackage[italian]{babel}
\usepackage{graphicx}%per le immagini
\usepackage{fancyhdr}
\usepackage[font = scriptsize, bf]{caption}
\usepackage[utf8x]{inputenc}
\usepackage[parfill]{parskip}
\usepackage{amsmath, amssymb}
\usepackage{moreverb}
\usepackage{algorithm}
\usepackage{algpseudocode}
\usepackage[usenames,dvipsnames]{color}
\usepackage[swapnames]{frontespizio}
\usepackage{url}
\usepackage{setspace}
\usepackage{eqparbox,array}
%\usepackage[subfigure]{tocloft}
\usepackage{amsthm}
\usepackage{amsmath}
\usepackage{url} % per scrivere gli indirizzi Internet
\usepackage{hyperref}
\usepackage{amsmath}
\usepackage{wrapfig}
\usepackage{caption}
\usepackage{subcaption}
%\newcommand{\algorithmiccomment}[1]{  //\emph{\textcolor{Gray}{#1}}}


% Sistema i margini per lasciare più spazio nella zona di rilegatura
\addtolength{\oddsidemargin}{+1,0cm} 
\addtolength{\evensidemargin}{+1,0cm} 
\onehalfspacing

% Imposta lo stile della prima pagina del capitolo
\fancypagestyle{plain}
{
    \fancyhead{}
    \fancyfoot[LE,RO]{\thepage}
    \renewcommand{\headrulewidth}{0pt}
}

\DeclareMathOperator*{\argmax}{arg\,max}
\newcommand{\compInterfacciaDB}{Data Interface}
\newcommand{\compLoader}{Loader}
\newcommand{\compMatrix}{Matrix Creator}
\newcommand{\compTermsSel}{Terms Selector}
\newcommand{\compPosition}{Position Calculator}
\newcommand{\compClustering}{Clustering Component}
\newcommand{\compEvolution}{Evolution Discoverer}

\hyphenation{ti-me-win-dow}

\graphicspath{{./immagini/}}


\begin{document}

\theoremstyle{definition}
\newtheorem{definizione}{definizione}
\newtheorem{teorema}{Teorema}
% Imposta lo stile di intestazione e piè di pagina della parte iniziale
	\pagestyle{fancy}
	\fancyfoot{}
	\fancyfoot[LE,RO]{\thepage}
	\fancyhead{}
	\renewcommand{\headrulewidth}{0pt}
	\headheight = 15pt


	% frontespizio
	\begin{frontespizio}
		\Universita{Bari - ``Aldo Moro''}
		\Logo[3.5cm]{logo_uni}
		\Facolta{Scienze MM.FF.NN.}
		\Corso{Informatica}
		\Annoaccademico{2014-2015}
		\Titoletto{Tesi di laurea\\in\\Basi di conoscenza e Data Mining}
		\Titolo{TITOLO TESI}
		\Candidato[573394]{Rocco Caruso}
		\NCandidato{Laureando}
		\Relatore{Chiar.mo Prof. Michelangelo Ceci}
		\Relatore{Chiar.mo Prof. Donato Malerba}
		\Correlatore{Dott.ssa Fabiana Lanotte}
		\Correlatore{Dott.    Roberto Corizzo}
		\Margini{3cm}{2cm}{2cm}{2cm}
	\end{frontespizio}
	\frontmatter
	
	% dedica
	\null\vspace{\stretch{1}}
		\begin{flushright}
			\emph{Ahi mè ...}
		\end{flushright}
	\vspace{\stretch{2}}\null
	
	% indice
	\tableofcontents
	\listoftables
	\listoffigures
	\newpage
	\color{white}
	a
	\color{black}
%******************************************************************
% Materiale iniziale
%******************************************************************
%% !TEX encoding = UTF-8
% !TEX TS-program = pdflatex
% !TEX root = ../Tesi.tex
% !TEX spellcheck = it-IT

%*******************************************************
% Frontespizio
%*******************************************************
\begin{frontespizio}
\Preambolo{\usepackage{iwona}} % riga da commentare se non si carica ArsClassica

\Universita{Bologna}
\Logo{Sigillo}
\Facolta{Teologia}
\Corso{Belle Lettere}
\Annoaccademico{2011--2012}
\Titoletto{Tesi di laurea magistrale}
\Titolo{La mia tesi: \\ la prova ontologica \\ dell'esistenza di Dio}
\Sottotitolo{Alcune considerazioni mutevoli}
\Candidato[AB123456]{Lorenzo Pantieri}
\Relatore{Enrico Gregorio}
\Relatore{Claudio Beccari}
\Correlatore{Tommaso Gordini}
\Correlatore{Ivan Valbusa}
\end{frontespizio}





%*******************************************************
% Frontespizio alternativo
%*******************************************************
%\begin{titlepage}
%\pdfbookmark{Frontespizio}{Frontespizio}
%\changetext{}{}{}{((\paperwidth - \textwidth) / 2) - \oddsidemargin - \hoffset - 1in}{}
%\null\vfill
%\begin{center}
%\large
%\sffamily
%\bigskip

%{\LARGE\myName} \\

%\bigskip

%{\Huge\myTitle \\
%}

%\bigskip
    
%\vspace{9cm}

%\begin{tabular}{cc}
%\parbox{0.3\textwidth}{\includegraphics[width=2.5cm]{Sigillo}}
%&
%\parbox{0.7\textwidth}{{\Large\myDegree} \\ 

%					{\normalsize
%					Relatore: \myProf \\
%%					Co-relatore: \myOtherProf \\
%					
%					\myUni \\
%					\myFaculty \\
%					\myDepartment \\
%					\myTime}}
%			\end{tabular}
%\end{center}
%\vfill
%\end{titlepage}







%\input{MaterialeInizialeFinale/Colophon}
%\input{MaterialeInizialeFinale/Dedica}
%\input{MaterialeInizialeFinale/Indici}
%\input{MaterialeInizialeFinale/Sommario+Abstract}
%\input{MaterialeInizialeFinale/Ringraziamenti}
% !TEX encoding = UTF-8
% !TEX TS-program = pdflatex
% !TEX root = ../tesi.tex
% !TEX spellcheck = it-IT

%*******************************************************
% Introduzione
%*******************************************************
\cleardoublepage
\chapter*{Introduzione}
Twitter è ad oggi il servizio di micro-blogging più utilizzato in assoluto, con circa 284 milioni di utenti attivi al mese, ogni giorno vengono prodotti oltre 500 milioni di tweets. Gli utenti di Twitter, possono pubblicare degli status, o \emph{tweets}, non on più lunghi di 140 caratteri. Se questo vincolo da una parte costituisce un forte limite, dall'altro rappresenta una delle caratteristiche fondamentali di Twitter: \emph{l'immediatezza}. Questo limite costringe gli utenti ad produrre messaggi molto sintetizzati, quasi come slogan, che quindi sono più facili da diffondere. Grazie a queste caratteristiche, qualsiasi persona che assiste o è coinvolta in un \emph{evento}, è in grado di diffondere informazioni in \emph{real-time}.
Talvolta, i tweets (Twitter microblog posts) diffondono notizie anche più velocemente dei media tradizionali (come la morte di Micheal Jackson \footnote{ttp://www.dailymail.co.uk/sciencetech/article-1195651/How-Michael-Jacksons-death-shut-
Twitter-overwhelmed-Google–killed-Jeff-Goldblum.html}).
Bisogna però sottolineare che questi tweet che possono riflettere eventi, rappresentano solo una piccola percentuale di tutti i tweet prodotti. La maggior parte infatti, è costituita da status personali, messaggi anche privi di senso, spam. Risulta quindi necessario un sistema che sia capace di scoprire "eventi o topics" da questo flusso di dati. Scoprire nuovi eventi o topics da Twitter, non è affatto un task banale sia 
sia la mole dei dati (oltre 400 milioni di tweet giornalmente), che per la natura stessa dei tweets. Se da un lato il limite a 140 caratteri ne rende più semplice la diffusione on-line, dall'altro complica ulteriormente il task, poiché gli utenti spesso,proprio a causa di tale limite,ricorrono a slang, vocaboli OOV\footnote{Out Of Vocabolary} o emoticons.
\section*{Twitter}
Twitter è ad oggi il servizio di "microblogging" più diffuso e con il più alto tasso di crescita.  Negli anni la sua popolarità crescente ha anche attirato anche un alto numero di ricercatori, come si può notare dall'alto numero di articoli riguardanti Twitter che sono stati pubblicati in numerosi campi di ricerca.
Sebbene il termine  “microblog” spesso possa indurre intendere tale servizio come una versione "micro" di un blog, sono due due media molto diversi fra loro \cite{export:69500}. I blog infatti sono progettati, principalmente per permettere ad utenti, di fornire commenti e opinioni su topic di cui sono esperti, gli autori devono anche garantire una certa validità dei contenuti. Dall'altra parte, i microblog, come Twitter, sono invece pensati per permettere di condividere opinioni, news, ma in maniera molto concisa (max 140 caratteri) proprio per far sì che si abbia una  diffusione tempestiva delle informazioni. Proprio grazie a questa caratteristica, i tweets possono essere pubblicati mediante dispositivi mobili, consentendo a chiunque sia testimone di un qualsiasi evento di diffondere la notizia in real-time. Twitter inoltre è anche un servizio di social networking, ogni utente può ricevere gli aggiornamenti ("follow")  di altri utenti senza previa approvazione. Questa relazione è asimmetrica e può essere concettualizzata come una Directed social network o \emph{follower network} 
\subsection*{Twitter come fonte di informazione}
Molte notizie sono state diffuse su Twitter anche prima della diffusione sui media classici. Uno degli esempi più significati è stato rappresentato dalla notizia della di Michael Jackson del 2009. Alle 2:26pm
del 24 Giungo 2009, la notizia trapelò su Twitter e fu diffusa in una maniera così virale che che Google la identificò come un attacco hacker.t . La validità della notizia fù verificata da Google solo 25 minuti dopo,   solo allora i media mainstream iniziarono a far diffondere la notizia \footnote{ttp://www.dailymail.co.uk/sciencetech/article-1195651/How-Michael-Jacksons-death-shut-
Twitter-overwhelmed-Google–killed-Jeff-Goldblum.html}.Anche nel caso del terremoto in Abruzzo del 6 aprile 2009, gli utenti Twitter hanno segnalato la notizia prima dei media tradizionali. 

%\pagestyle{scrheadings} 
%\cleardoublepage
%******************************************************************
% Materiale principale
%******************************************************************


% capitoli
	% frontespizio
	
	\frontmatter	
	\mainmatter

	% Imposta lo stile di intestazione e piè di pagina dei capitoli
	\fancyfoot{}
	\fancyhead{}
	\fancyhead[LE,RO]{\slshape \leftmark}
	\fancyfoot[LE,RO]{\thepage}
	\renewcommand{\headrulewidth}{1pt}
	\renewcommand{\chaptermark}[1]{%
	\markboth{\thechapter.\ #1}{}}


\chapter{Stato dell'Arte}
\label{cap:capitolo1}
% !TEX encoding = UTF-8
% !TEX TS-program = pdflatex
% !TEX root = ../Tesi.tex
% !TEX spellcheck = it-IT

%************************************************

%************************************************

\section{Event detection nei media tradizionali}
L'attività di event-detection è stata a lungo utilizzata per individuare eventi da stream testuali derivanti dai media più tradizionali come giornali o radio, infatti l'event-detection è stata per molto tempo oggetto di ricerca del programma di \emph{Topic Detection and Tracking TDT} \cite{Allan:2002:TDT:772260}, un' iniziativa	promossa dalla DARPA \footnote{Agenzia di ricerca agenzia per i progetti di ricerca avanzata per la difesa}, con lo scopo di organizzare stream di notizie testuali sulla base degli eventi di cui discutono. Secondo il TDT, l'obiettivo dell'attività di \emph{event detection}, è scoprire nuovi  eventi o eventi precedentemente non noti, a partire da stream di notizie testuali derivanti dai media tradizionali come notiziari o newswire, dove ciascun evento è definito come segue:


\begin{definizione}[Evento]
\label{def:evento}
\emph{: qualcosa, non banale, che accade in un luogo e tempo specifico}
\end{definizione}


Le tecniche di event-detection possono essere classificate in due macro categorie: \emph{document-piovot} e \emph{document piovot} a seconda che utilizzino feature dei documenti o feature temporali delle singole keywords presenti nei documenti. 
La prima scopre eventi effettuando un clustering dei documenti sulla base di una qualche funzione di distanza fra i documenti stessi \cite{Yang:1998:SRO:290941.290953}, mentre nella seconda si studia la distribuzione delle singole parole e scoprono nuovi eventi raggruppando le parole \cite{Kleinberg:2002:BHS:775047.775061} 
Come evidenziato da \cite{Yang:1998:SRO:290941.290953} infatti, l'event detection può essere ricondotto al problema della scoperta di pattern in uno stream testuale, quindi il modo più naturale per scoprire nuovi eventi, è quello di usare un algoritmo di clustering. Il task di event-detection si può suddividere in tre fasi principali: data preprocessing, data rappresentation, data organization o clustering. Nella fase di preprocessing vengono applicate al testo delle classiche tecniche di NLP come la rimozione di stopwords, tokenizzazione e stemming.
I modelli di rappresentazione di dati più utilizzati per l'event detection sono  \emph{il modello vettoriale} e  \emph{il modello bag of words}, i cui elementi saranno diversi da zero, se il termine corrispondente è presente nel documento. A ciascun termine nel vettore, è assegnato un peso secondo lo schema \emph{tf-idf}\cite{Salton:1989:ATP:77013} che valuta quanto è importante una parola per un documento all'interno di un corpus. Questo modello di rappresentazione non prende in considerazione l'ordine temporale delle parole ne le caratteristiche sintattiche o semantiche del testo come il part of speech tag o named entities. Per questa ragione utilizzando questo modello, ad esempio, sarebbe difficile distinguere due eventi simili ma accaduti ad un mese di distanza fra loro. Nel lavoro di \cite{Yang:1998:SRO:290941.290953} il task di scoperta di nuovi eventi da uno stream testuale di news è suddiviso in due fasi principali:
Retrospective Event Detection (RED), New Event Detection (NED). La prima fase (RED) comporta la scoperta di eventi da una collezione già nota di documenti, mentre nella seconda si cerca di identificare gli eventi dallo stream di notizie in tempo reale. Per il RED è stato utilizzato un algoritmo di clustering gerarchico: \emph{Group Average Clustering GAC}, che consente anche di descrivere gli eventi identificati con diversi livelli di granularità. Per la fase di New Event Detection, invece, solitamente viene adottato un agoritmo di clustering incrementale single-pass \cite{Allan:2002:TDT:772260,Yang:1998:SRO:290941.290953} che consente di suddividere i documenti nei vari cluster non appena arrivano dallo stream. In particolare ciascun documento viene elaborato sequenzialmente e viene assorbito dal cluster più simile, o verrà creato un nuovo cluster se la similarità è al di sotto di una soglia prestabilita. In un ambiente di detection on-line (NED) un forte vincolo è costituito dal fatto che non si può hanno informazioni di eventi futuri,ovvero non è possibile utilizzare dati provenienti da documenti successivi, cronologicamente, a quello corrente. Utilizzando un modello di rappresentazione vettoriale, questo vincolo pone delle problematiche su come gestire la crescita del vocabolario dei termini quando vengono aggiunti nuovi documenti al corpus e come modificare delle statistiche inerenti l'intero corpus come l'IDF. La soluzione suggerita da  \cite{Yang:1998:SRO:290941.290953} è quella di modificare il vocabolario dei termini in maniera incrementale e modificare l'IDF ogni qual volta viene aggiunto un nuovo documento. 
\begin{equation}
\label{eq:incIDF}
idf_t(w)=log_2 \left(\frac{N_t}{df_t(w)} \right) 
\end{equation}
dove $N_t$ è il numero di documenti fino al tempo $t$ e $df_t(w)$ è la document frequency della keyword $w$ fino al tempo $t$.
In pratica questi approcci NED, tendono a divenire molto costosi sia in termini di risorse computazionali che di tempo richiesto, e in taluni casi addirittura irrealizzabili se non utilizzando delle tecniche che ne migliorino l'efficienza. Una possibile tecnica per ridurre i costi è quella di utilizzare una \emph{time window} \cite{Luo:2007:RRN:1247480.1247536,Papka:1999:ONE:897559} per limitare il numero vecchi documenti da analizzare quando si prende in considerazione un nuovo documento. Utilizzare una finestra temporale non solo riduce i costi, ma permette anche di limitare lo scope degli eventi scoperti, consentendo di identificare eventi simili ma che accadono in uno slot temporale diverso \cite{Yang:1998:SRO:290941.290953}. Tutte queste tecniche per il TDT si basano sull'assunzione che tutti i documenti siano rilevanti e contengono informazioni di eventi, poichè lavorano su stream di informazioni affidabili,  assunzione che è chiaramente violata  per quanto riguarda lo stream di Twitter.
Nelle tecniche feature-pivot un evento viene invece modellato come una attività che presenta picchi di frequenza (burst), ovvero un evento è rappresentato dall'insieme di keywords che presentano un burst \cite{Allan:2002:TDT:772260}. L'assunzione fatta da queste tecniche è che alcune parole avranno un incremento di utilizzo repentino quando accade un evento. Nel lavoro di \cite{Kleinberg:2002:BHS:775047.775061} viene utilizzato un'automa a stati infiniti per poter identificare i burst delle keyword all'interno dello stream testuale. Gli stati dell'automa corrispondono alla frequenze delle singole parole, mentre le transizioni fra a gli stati identificano i burst che corrisopondo a un  cambiamento significativo nella frequenza.
A differenza delle tecniche document-pivot, in questo caso si cercano di identificare eventi raggruppando (ovvero effettuando il clustering)   quelle keyword che presentano un burst, piuttosto che i documenti. Nel lavoro di \cite{Allan:2002:TDT:772260} la frequenza delle parole viene modellata tramite una distribuzione bionomiale, dpoi vengono idnvidiuate le bursty-keywords sulla base di una soglia euristica, per poi raggrupparle al fine di identificare gli eventi.
\section{Event Detection in Twitter}
L'attività di Event Detection nei microblogs come Twitter, è concettualmente molto simile all' Event Detection nei media tradizionali. In entrambi i casi, viene dato in input al sistema  uno stream di documenti testuali e l'obiettivo è quello di scoprire degli eventi raggruppando i documenti o le singole parole contenute nei documenti stessi. L'unica differenza che hanno è il tipo e il volume di documenti dello stream che devono analizzare, in pratica tuttavia, questa unica differenza si riflette in una serie di nuove sfide per il task dell'event detection. 
Innanzitutto il volume di documenti nel caso dei microblogs come twitter  è di diversi ordini di grandezza più grande rispetto ai media tradizionali, ma soprattutto nel caso di stream derivanti dai media tradizionali, tutti i documenti hanno una qualche rilevanza rispetto ad un avvenimento, una notizia. Nel caso dei tweet, invece, vi possono essere grandi quantità di messaggi privi di significato (pointless babbles) \cite{DBLP:conf/icwsm/HurlockW11} e rumors \cite{Castillo:2011:ICT:1963405.1963500}. Inoltre le caratteristiche di Twitter e la sua poplarità sono molto allettanti per spammers e altri e altri content polluters \cite{DBLP:conf/icwsm/LeeEC11} per disseminare pubblicità, virus, pornografia pishing o anche per compromettere la reputazione del sistema. La sfida più grande che bisogna affrontare nell'attività di event detection per i tweet, è quindi quella di poter separare informazioni mondane e inquinate da informazioni su eventi reali. Altre difficoltà sono causate principalmente dalla brevità dei messaggi (max 140 caratteri), dall'uso di abbreviazioni, errori di spelling e grammaticali, e l'uso improprio della struttura delle frasi e l'utlizzo di più lingue nel medesimo tweet. Per queste ragioni anche le tecniche tradizionali di natural language processing  meno appropriate per i tweet. Un altro lavoro interessante che si colloca in quest'area è TwitterStand \cite{Sankaranarayanan:2009:TNT:1653771.1653781}: un sistema per scoprire le ultime  notizie da twitter. Per poter distinguere il rumore dalle news, hanno selezionato manualmente 2000 utenti come "Seeders" ovvero utenti che pubblicano su Twitter news come stazioni televisive, giornali, bloggers etc. I tweets non appartenenti a questi seeders, invece, vengono filtrati per mezzo di un classificatore Naive Bayes. Dopo aver filtrato i tweet, viene applicato un algoritmo di clustering incrementale al fine di creare cluster tali che ognuno corrisponda ad una "news". Il modello di rappresentazione utilizzato è quello vettoriale con pesatura tf-idf e la funzione di similarità adottata è quella del coseno.  Inoltre viene mantenuta una lista di cluster "attivi" per ridurre il numero di confronti da effettuare. In particolare un cluster viene definito inattivo se la media delle date di pubblicazione dei tweet, non supera i tre giorni. Una volta identificati i topic (news), il sistema cerca di localizzare ciascun cluster, ovvero cerca di assegnare una posizione geografica sia sulla base del contenuto testuale che sui geotag presenti nei tweet.
Un altro sistema per scoprire news da twitter è stato proposto Phuvipadawat e Murata \cite{Phuvipadawat:2010:BND:1913791.1913911}. In questo lavoro per ridurre il rumore, i tweet vengono innanzitutto campionati utilizzando attraverso le  streaming API di twitter e fornendo delle specifiche keyword da monitorare (\#breakingNews, \#breaking \#news). I tweet raccolti vengono sucessivamente indicizzati tramite Apache Lucene\footnote{https://lucene.apache.org/core/}. I tweet simili fra loro vengono raggruppati per poter identificare news. Anche in questo caso la similarità adottata si basa sulla similarità del coseno fra le rappresentazioni tf-idf dei tweet, ma viene assegnato un \emph{boost} per quei termini che corrispondono a nomi propri e per hashtag e username. I nomi propri sono identificati utilizzando lo Stanford Name Entity Recognizer \footnote{http://nlp.stanford.edu/software/CRF-NER.shtml} addestrato su un corpora di news tradizionali.
 I nuovi messaggi saranno inclusi in un cluster se sono simili al primo tweet   e ai top-k termini  presenti nel cluster. I cluster prodotti vengono poi ordinati sulla base 
dell'affidabilità (numero di followers) e popolarità (numero di retweet).
Gli autori hanno fortemente sottolienato l'importanza dell'identificazione dei nomi propri al fine di migliorare il calcolo della similarità fra i tweet, e di conseguenza migliorare l'accuratezza generale del sistema.
Nel lavoro di \cite{DBLP:conf/icwsm/BeckerNG11} viene posta maggiore attenzione sull'identificazione di eventi reali da Twitter. Il metodo da loro proposto utilizza un algoritmo di clustering incrementale, che raggruppa i tweet simili fra loro e poi classifica i risultanti cluster in eventi real-world o non events. L'algoritmo di clustering utilizzato è il classico algoritmo di incrementale basato su soglia, ogni tweet è rappresentato mediante un boosted tf-idf vector, e viene utilizzata la similarità del coseno per valutare la distanza fra un tweet e il centroide di ogni cluster. Oltre ai classici step di pre-processing come tokenization, stop-word removal e stemming, viene raddoppiato il peso per gli hashtag, poiché sono considerati fortemente indicativi del contenuto del messaggio. Gli autori hanno definito quattro tipologie di feature per i cluster individuati, per poter distinguere eventi reali  da non-event o "twitter center topic" :
\begin{enumerate}
\item temporal-features: sono state definite un insieme di caratteristiche temporali per poter caratterizzare il volume dei termini più frequenti all'interno di un cluster.
\item social-features: insieme di feature che caratterizzano il grado di interazione degli utenti nei tweet del cluster come la percentuale di retweet, di replies. L'assunzione fatta è che i cluster contenenti un alta percentuale di retweet potrebbero non contenere informazioni di eventi reali.
\item topical-features: descrivono la coerenza del cluster rispetto ad un topic. L'idea sottostante è che i cluster relativi ad eventi tendono a svilupparsi attorno ad un tema comune, al contrario di non-event cluster che si sviluppano attorno a diversi termini comuni e.g: ("sleep" or "work"). Per stimare questa coerenza, gli autori calcolano la media della similarità dei tweet rispetto al centroide.
\item twitter centric-features: queste caratteristiche hanno lo scopo di identificare le attività twitter-centric. Esempi di queste feature sono la percentuale di tweet contenenti hashtag e la percentuale di tweet contenenti l'hashtag più utilizzato. Gli autori presumono che un'alta percentuale della prima stia ad indicare un topic conversazionale.
\end{enumerate}
Poichè i cluster evolvono nel tempo, queste feature sono periodicamente aggiornate.
Sulla base di queste feature, è stato addestrato una support vector machine (SVM) a partire da un insieme di cluster etichettati, che verrà usata per decidere se un nuovo cluster contiene o meno informazioni relative a eventi reali. AL fine di permettere di eseguire il task di scoperta di eventi su larga scala invece, in \cite{Petrovic:2010:SFS:1857999.1858020} è stato ideato un algoritmo che fa ricorso al \emph{Local Sensitive Hashing}\cite{Lsh}  grazie a cui è possibile superare alcuni limiti degli approcci classici.
L'utilizzo dell'LSH, permette di ridurre drasticamente il tempo richiesto per la scoperta del Nearest Neighbor di un punto in uno spazio vettoriale.
In particolare per ogni documento proveniente dallo stream, viene calcolata la distanza rispetto al suo nearest neighbor, che verrà adoperata sia per determinare il grado di novità del documento rispetto ai precedenti (Novelty Score), sia per dall'altro è utilizzato per creare dei "thread di tweets" o in altri termini cluster, che come negli altri metodi document-pivot precedentemente descritti, si assume che  corrisponderanno ad eventi. Viene definita una relazione \emph{links} come segue: si dice che un tweet   $a$ ha un $link$ verso il tweet $b$  se la loro distanza è al di sotto di uuna soglia.  
 \begin{equation}
  a \ links \ b \Longleftrightarrow 1-cos(a,b)<t
\end{equation}
Dopodiché ciascun tweet $a$  verrà  assegnato ad un thread esistente se la sua distanza dal suo nearest neighbor è al di sotto la soglia, o sarà considerato come il primo tweet di un nuovo thread. Nel primo caso il tweet $a$ sarà assegnato al medesimo thread cui appartiene il suo nearest neighbor. Cambiando la soglia $t$ 
è possibile controllare la granularità dei thread prodotti. Se $t$ è un valore molto alto, si avranno pochi thread ma molto grandi, mentre un valore di $t$ molto basso genererà molti thread di piccole dimensioni.
Una volta individuati i thread, vengono considerati solo quelli che crescono più velocemente. Un alto fattore di crescita da infatti indicazione del fatto che la notizia del nuovo evento si stia diffondendo,  per tale ragione vengono restituiti solo i threads con il grow-rate più alto. L'aspetto più interessante del lavoro di Petrovi\'c et al. \cite{Petrovic:2010:SFS:1857999.1858020} è rappresentato dal fatto che il loro sistema riesce a processare un nuovo documento proveniente dallo stream, in tempo e spazio costante.  Per ottenere tale risultato non solo impiegano l'lsh per ridurre il numero di confronti da effettuare, ma impongono ulteriori limiti per applicare tali tecniche allo streaming dei tweet. Infatti poiché il numero di bucket dell lsh sebbene possa essere alto è un numero finito, e in uno stream di dati, il numero di documenti che possono ricadere in un bucket potrebbe crescere a dismisura. Per tale ragione pongono un limite sul numero di documenti che può contenere un singolo bucket. Quando un bucket raggiungerà la sua massima capienza, verrà eliminato il documento più vecchio. Questa restrizione sebbene renda il numero di confronti costante, tale costante può divenire piuttosto alta e quindi viene posto un ulteriore vincolo per limitare superiormente il numero di confronti. 
Bisogna sottolineare però, che se da un lato questi vincoli permettono al sistema di lavorare in setting incrementale, dall'altro andranno ad esacerbare il problema della \emph{fragmentation}, poiché il sistema effettuando per ogni nuovo tweet un numero costante di confronti tenderà a generare un numero maggiore di sotto-eventi. La frammentazione si presenta quando tweets che discutono dello stesso evento sono assegnati a cluster diversi. Questo problema affligge la maggior parte tutti i sistemi che tentano di estrarre documenti dallo streaming fi tweet, tramite il clustering. In particolare impostando una soglia di similarità troppo bassa, si avrà una maggiore probabilità di frammentazione, impostando un valore molto basso invece, può dare origine al problema opposto il \emph{merging} ovvero il problema duale della fragmentation, che si presenta quando tweet di che discutono di eventi diversi sono raggruppati in un unico cluster. Per alleviare il problema della frammentazione, Sankaranarayanan et al \cite{Sankaranarayanan:2009:TNT:1653771.1653781} suggeriscono, una volta identificati i cluster, di tentare di fondere periodicamente i cluster simili o duplicati. \'E doveroso sottolineare che tutti i metodi descritti precedentemente, si basano in linea di massima   sulla mera similarità testuale fra i tweet,  di conseguenza non riusciranno ad identificare pattern di tweet composti da parole sintatticamente diverse ma  semanticamente correlate (sinonimia), o al contrario produrranno cluster di tweet composti da termini identici ma, semanticamente non correlati (polisemia).



\chapter{Unsupervised Learning}
\label{cap:capitolo2}
\input{Capitoli/Capitolo2}
\chapter{Sperimentazione}
\label{cap:capitolo4}
% !TEX encoding = UTF-8
% !TEX TS-program = pdflatex
% !TEX root = ../Tesi.tex
% !TEX spellcheck = it-IT

%************************************************

%************************************************
In questo capitolo si descriverà la progettazione e l'esecuzione della sperimentazione. Si partirà pertanto dai dati su cui quest'ultima è stata effettuata, proseguendo con la scelta delle modalità di esecuzione più interessanti e concludendo con una serie di tabelle e grafici contenenti i risultati ottenuti, opportunamente commentati.  
\section{Sperimentazione modulo estrazione dei perpetratori}
L'obiettivo di questa prima sperimentazione è valutare l'efficacia del sistema analizzato al fine di comprendere le cause di un eventuale successo o insuccesso delle tecniche utilizzate e quindi decidere se proseguire gli studi in questa direzione o definire strategie alternative.

La sperimentazione del sistema è stata effettuata sfruttando i dati appartenenti al database GTD (Global Terrorism Database)\cite{GTD}.
Questo database open-source, reso disponibile dal National Consortium for the Study of Terrorism and Responses to Terrorism (START), contiene informazioni su eventi terroristici avvenuti nel mondo nel periodo 1979-2011. A differenza di altri database, GTD contiene dati su incidenti terroristici nazionali e internazionali, per un totale di 104.000 istanze. Per ogni incidente sono disponibili una serie di informazioni quali data, luogo dell'incidente, armi utilizzate, numero di vittime e nomi degli individui o gruppi responsabili.
Si descrivono di seguito le caratteristiche principali del database GTD:
\begin{itemize}
	\item Contiene informazioni su 104.000 attacchi terroristici;
	\item Ad oggi la più completa base di dati non classificata di eventi terroristici avvenuti nel mondo;
	\item Include informazioni su più di 47.000 bombardamenti, 14.000 assassinii e 5300 rapimenti dal 1979 ad 2011;
	\item Contiene per ogni caso almeno 45 variabili, fino ad arrivare a 120 variabili nel caso di incidenti recenti;
	\item Supervisionato da un gruppo consultivo di 12 esperti in campo terroristico;
	\item Oltre 3.500.000 articoli di news e 25.000 sorgenti di news sono stati analizzati per raccogliere dati sugli incidenti avvenuti nel solo periodo 1998-2011.
\end{itemize}

Per la sperimentazione sono stati selezionati 10732 istanze (ognuna associata a uno su 50 distinti nomi di perpetratori), da cui sono state estratti la descrizione testuale dell'incidente terroristico (news), la data associata all'evento, titolo dell'articolo e il nome dei perpetratori, utilizzati per popolare il database di TB-CREDIS. 
Delle 10732 istanze solo 7361 sono utilizzate per l'estrazione delle entità nominali (per un totale di 44 nomi di perpetratori), in quanto la descrizione testuale delle istanze eliminate non conteneva il nome dei perpetratori, quindi sarebbe stato impossibile per qualsiasi Named Entity Recognizer individuare le entità corrette.

\paragraph{} Tratteremo di seguito come valutare un sistema di Entity Extraction, in modo da verificare se il sistema realizzato ha prodotto risultati significativi. 

In generale la valutazione dei sistemi di machine learning è effettuata sperimentalmente piuttosto che analiticamente. Questa infatti non può essere formalizzata (a causa della sua natura soggettiva) e quindi non può essere valutata analiticamente.
La valutazione sperimentale solitamente misura l'efficacia di una sistema, ossia l'abilità nel prendere la giusta decisione. Questa viene
misurata in termini di \textit{precisione} (precision) e  \textit{richiamo} (recall).
In particolare, definite le seguenti misure:
\begin{itemize}
	\item \textbf{True Positive}: entità correttamente etichettate 
	
	
	(\textit{President \textbf{Bush} attended the ceremony});
	\item \textbf{True Negative}: termini correttamente non etichettati come positivi 
	
	(\textit{ \texttt{President} \textbf{Bush} attended the ceremony})
	\item \textbf{False Positive}: termini non entità, erroneamente etichettati come tali
	
	(\textit{President \textbf{Bush} attended the \textbf{ceremony} })
	\item \textbf{False Negative}: entità erroneamente non etichettate come tali
	
	(\textit{President Bush attended the ceremony.})
\end{itemize}
è possibile calcolare una stima della precisione (ossia il numero di entità correttamente etichettate sul numero totale di entità individuate), richiamo (ossia il numero di entità correttamente etichettate rispetto a tutte le entità presenti) e F-Score (che permette di definire un buon trade-off tra precisione e richiamo) del sistema:
\begin{equation}
\label{recall}
Recall= \frac{True Positive}{(True Positive + False Negative)};
\end{equation}
\begin{equation}
\label{precision}
Precision= \frac{True Positive}{(True Positive + False Positive)};
\end{equation}
\begin{equation}
\label{Fscore}
F-Score= \frac{2(Precision * Recall)}{(Precision + Recall)};
\end{equation}

Praticamente, per ottenere queste misure è possibile procedere in due modi:
\begin{itemize}
	\item Micro-Media, i valori delle misure sono valutati localmente alle classi e poi mediati sul loro numero per ottenere una stima globale delle
prestazioni del sistema;
	\begin{equation}
	 Precision= \frac{TP}{TP+FP}= \frac{\sum_d TP_{d}}{\sum_d (TP_d+FP_d)}
	\end{equation}

	\begin{equation}
	 Recall= \frac{TP}{TP+FN}= \frac{\sum_d TP_{d}}{\sum_d (TP_d+FN_d)}
	\end{equation}
	dove $TP_d, FP_d$ e $FN_d$ sono rispettivamente il numero di entità positive correttamente etichettate (TP), il numero di termini erroneamente etichettati come entità (FP) e il numero di entità non etichettate come tali (FN) per il generico documento $d$ appartenente alla collezione.
	\item Macro-Media, la precisione e il richiamo sono prima valutate localmente per ogni documento $d$ e poi globalmente dalla media dei risultati:
	
	\begin{equation}
	Precision= \frac{\sum_d Precision_d}{|Documents|}
	\end{equation}
	
	\begin{equation}
	Recall= \frac{\sum_d Recall_d}{|Documents|}
	\end{equation}
\end{itemize}
dove $Documents$ è l'insieme dei documenti appartenenti alla collezione, mentre $Precision_d$ e $Recall_d$ sono la precisione e il richiamo calcolati sul documento $d \in Documents$.


Per valutare le prestazioni del sistema realizzato, i risultati ottenuti verranno confrontati con la precisione e il richiamo calcolati sul solo sistema di Named Entity Extraction(NER), che verrà quindi utilizzato come baseline. 
Per l'estrazione delle entità verranno utilizzati i tool TextPro e LingPipe, mentre per la creazione dell'albero delle dipendenze verrà utilizzato Stanford Parser. 
Per ogni documento verranno calcolate le misure precedentemente definite utilizzando la seguente strategia.

Per ogni documento:
\begin{itemize}
\item \textbf{True Positive}=1 se il sistema ha restituito almeno 1 volta il perpetratore, 0 altrimenti;
\item \textbf{False Negative}=1 se il sistema non ha mai restituito il perpetratore, 0 altrimenti;
\item \textbf{False Positive} = numero di entità estratte - numero di entità perpetratici.
\end{itemize}


Calcolati i True Positive, False Positive, True Negative, False Negative si realizza il calcolo della precisione e richiamo tramite micro-media.
Si riportano pertanto le caratteristiche hardware e software utilizzate per questa sperimentazione:
\begin{itemize}
	\item CPU Intel Core i5 2500K - Core: 4 - Frequenza: 4Ghz;
	\item RAM 8GB DDR3 1600Mhz;
	\item Hard Disk Western Digital 500GB 5400rpm;
	\item S.O. Microsoft Windows 7 64bit;
	\item DBMS PostgreSQL 9.2;
	\item TextPro 1.4;
	\item LingPipe Royalty Free;
	\item Wordnet 2.1;
	\item Strawberry Perl (64-bit) 5.16.2.1;
	\item Java 1.7 update 9 - 64bit.
\end{itemize}

Per rendere compatta la presentazione dei risultati, saranno utilizzate le seguenti abbreviazioni:
\begin{itemize}
\item \textbf{TPner} = Named Entity Extraction realizzata con TextPro;
\item \textbf{LPner} = Named Entity Extraction realizzata con LingPipe;
\item \textbf{TP+Stanford} = estrazione delle entità realizzata con TPner e selezione dei perpetratori attraverso Stanford Parser;
\item \textbf{LP+Stanford} = estrazione delle entità realizzata con TPner e selezione dei perpetratori attraverso Stanford Parser;
\end{itemize}
Infine si riportano di seguito le tabelle e i grafici dei risultati ottenuti:
\begin{table}[H]
	\centering
	\footnotesize
	\begin{tabular}{|cccc|}
	\hline
	\textbf{Metodo}  & \textbf{Precisione} & \textbf{Richiamo} & \textbf{F-Score} \\ \hline
	TPner & 0,6 & 0,83 & 0,70\\ 
	LPner & 0,52 & 0,85 & 0,65\\ 
	TP+Stanford & 0,86 & 0,87 & 0,86\\ 
	LP+Stanford & 0,72 & 0,76 & 0,74\\ 
	\hline
	\end{tabular}
	\caption{Risultati sperimentazione estrazione perpetratori}
	\label{tabConfrontoRis}
\end{table}
\begin{figure}[H]
\centering
\subfigure{\includegraphics[scale=0.5]{Immagine39}}\qquad
\subfigure{\includegraphics[scale=0.5]{Immagine40}}
\subfigure{\includegraphics[scale=0.5]{Immagine41}}\qquad
\caption{Confronto tra Precisione, Richiamo e F-Score tra i quattro metodi}
\label{figConfrontoRis}
\end{figure}
Osservando la figura \ref{figConfrontoRis} e la tebella \ref{tabConfrontoRis} è possibile notare come la precisione e il richiamo maggiori sono ottenuti combinando le informazioni estratte da TextPro con le dipendenze sintattiche calcolate da Stanford Parser (precisione=86\%, richiamo=87\%).

Poiché il confronto tra le entità estratte dal Named Entity Recognizer e quelle estratte dal Dependencies Analyzer è realizzato tramite string matching è possibile che molte corrispondenze corrette vengano perse a causa di errori ortografici o tipografici.
Per esempio, dato il documento:

\textit{``A suicide bomber wounded two police officers and a bystander as he approached a police station in Diyarbakir, Turkey. The Kurdistan Worke?s Party (PKK) was suspected of this attack.''}, associato all'organizzazione criminale \textit{``Kurdistan Worke's Party''},

i sistemi Named Entity Recognizer e Dependencies Analyzer riconoscono la stringa ``Kurdistan Worke?s Party'' come entità perpetratrice, ma poichè non esiste un matching perfetto tra ``Kurdistan Worke's Party'' e ``Kurdistan Worke?s Party'' l'entità estratta è scartata dalla lista delle entità corrette.
Per risolvere questo problema, e quindi migliorare il richiamo dei sistemi, è possibile utilizzare funzioni di matching che verifichino l'esistenza di una corrispondenza parziale tra le entità predefinite e quelle estratte \cite{Deng12}.
Un altro problema da affrontare per migliorare la qualità dei risultati prodotti è rappresentato dall'esistenza di possibili gerarchie tra le entità criminali predefinite. 
Per esempio il dataset analizzato contiene come entità perpetratrici le organizzazioni ``Al-Qa`ida'', ``Al-Qa`ida in the Arabian Peninsula (AQAP)'', ``Al-Qa`ida in Iraq'', che sono caratterizzate da relazioni di tipo gerarchico. In questo caso il calcolo delle misure True Positive, False Positive, True Negative e False Negative, come precedentemente definito, non risulta appropriato in quanto gli errori di estrazione non hanno tutti pari importanza, ma  si possono distinguere errori \textit{più gravi}, consistenti  nell'estrazione di entità che non sono correlate gerarchicamente con quella di effettiva appartenenza, ed errori \textit{meno gravi} in cui le entità estratte sono correlate gerarchicamente rispetto a quella effettiva.

\section{Sperimentazione predizione attività criminali}
Per quanto riguarda la sperimentazione della seconda componente realizzata si è deciso di utilizzare dati generati artificialmente. Nel processo di generazione sono state fatte determinate assunzioni e si è dovuto tenere conto di alcuni fattori. 

In particolare, sono stati considerati i seguenti obiettivi:
\begin{itemize}
	\item individuazione di un insieme di finestre temporali di ampiezza plausibile, relativamente ai fenomeni che si stanno analizzando;
	\item definizione di un insieme di criminali;
	\item per ciascuno di essi, generazione di un insieme di documenti relativi ad un particolare argomento (crimine), con data/ora appartenente ad una delle finestre temporali definite;
	\item definizione di alcune evoluzioni nel tempo.
\end{itemize}

Riguardo alle finestre temporali, poiché i fenomeni che si immagina di analizzare sono piuttosto lenti, si è deciso di porre l'ampiezza ad 4 anni e di rappresentare il periodo 1970-2009 (tabella \ref{TAB_FinestreTemporali}).
\begin{table}[htb]
	\centering
	\footnotesize
	\begin{tabular}{|cccc|}
	\hline
	\textbf{tw\_id} & \textbf{tw\_name} & \textbf{tw\_start\_date} & \textbf{tw\_end\_date} \\ \hline
	1 & 1970-1973 & 1970-01-01 00:00:00 & 1973-12-31 23:59:59 \\
	2 & 1974-1977 & 1974-01-01 00:00:00 & 1977-12-31 23:59:59 \\
	3 & 1978-1981 & 1978-01-01 00:00:00 & 1981-12-31 23:59:59 \\
	4 & 1982-1985 & 1982-01-01 00:00:00 & 1985-12-31 23:59:59 \\
	5 & 1986-1989 & 1986-01-01 00:00:00 & 1989-12-31 23:59:59 \\
	6 & 1990-1993 & 1990-01-01 00:00:00 & 1993-12-31 23:59:59 \\
	7 & 1994-1997 & 1994-01-01 00:00:00 & 1997-12-31 23:59:59 \\
	8 & 1998-2001 & 1998-01-01 00:00:00 & 2001-12-31 23:59:59 \\
	9 & 2002-2005 & 2002-01-01 00:00:00 & 2005-12-31 23:59:59 \\
	10 & 2006-2009 & 2006-10-01 00:00:00 & 2009-12-31 23:59:59 \\
	\hline
	\end{tabular}
	\caption{Finestre temporali definite.}
	\label{TAB_FinestreTemporali}
\end{table}

Riguardo ai criminali, non essendo interessati ad identificarli personalmente, è possibile generarne un numero a piacimento, assegnando un identificatore numerico progressivo a ciascuno di essi.

La problematica fondamentale risiede nella generazione dei documenti, la quale non deve descrivere una situazione troppo banale ma, allo stesso tempo, neanche troppo particolare. Si è deciso di individuare un insieme di argomenti, ciascuno dei quali rappresentati da alcuni termini chiave:

\begin{itemize}
	\item \textbf{omicidio:} \emph{homicide, weapon, pistol, shotgun, gun, blood, dead, handgun, victim, knife, bullet, murder, assassin, killer};
	\item \textbf{droga}: \emph{drug, cocaine, heroine, hashish, traffic, import, hide, overdose, psychoactive, hallucinogen, ketamine};
	\item \textbf{corruzione:} \emph{corruption, money, contract, politic};
	\item \textbf{estorsione:} \emph{extortion, money, violence, organized, weapon, commercial};
	\item \textbf{pedofilia:} \emph{pedophilia, child, young, rape, sexual, porn, violence};
	\item \textbf{stupro:} \emph{rape, sexual, violence, blood};
	\item \textbf{rapina-furto:} \emph{robbery, rapine, weapon, money, steal, theft}.
\end{itemize}

In totale sono state previste dunque 7 tipologie di crimini. Una delle difficoltà che il sistema si trova ad affrontare è dunque quella di riuscire a differenziare tra le diverse tipologie, considerando la presenza di termini in comune a più argomenti, la presenza di termini non propriamente specifici di atti criminali (es. \emph{money, traffic}) e la differente quantità di termini che descrivono gli argomenti. Oltre ai termini specifici, i documenti sono stati completati da una serie di termini presi casualmente da un vocabolario (termini rumore). 

In generale, nella generazione dei documenti si è tenuto conto dei seguenti fattori: 
\begin{itemize}
	\item la quantità, per ciascun criminale;
	\item la lunghezza, quantificabile in numero di parole;
	\item l'argomento, che deve riguardare una delle 7 tipologie di crimine previste;
	\item la collocazione temporale.
\end{itemize}
Per la sperimentazione si è deciso di definire 4 dataset generati a partire da 500 criminali di Training e 100 di Test, ciascuno con al massimo 100 documenti associati, aventi data compresa tra l'1 Gennaio 1970 e il 31 Dicembre 2009. La probabilità per i criminali di Training di comparire in una delle 10 finestre temporali precedentemente definite è pari al 90\%, mentre i criminali di Test compariranno in tutte le finestre temporali.
 
Si riporta di seguito l'algoritmo generale.
\footnotesize
\begin{algorithm}[H]
\caption{Generazione dei documenti}
\begin{algorithmic}

	\State $numCrim\gets 600$;	\Comment{Numero di criminali da generare in totale}
	\State $numCrimTraining\gets 500$ \Comment{Numero di criminali di Training da generare}
	\State $numDocs\gets 100$;	\Comment{Numero massimo di documenti per ciascun criminale}
	\State $biasPercentage\gets 0.7$;	\Comment{Probabilità che un criminale commetta la stessa tipologia di crimini della finestra temporale precedente}
	\State $criminalTrainingProb\gets 0.9$;	\Comment{Probabilità che un criminale di Training abbia documenti in una finestra temporale}
	\State $criminalTestProb\gets 1$;	\Comment{Probabilità che un criminale di Test abbia documenti in una finestra temporale}
	\State $n$	\Comment{Numero di finestre temporali definite. In questo caso è pari a 10.}
	\State $topics$ \Comment{Insieme degli argomenti.}
 	
 	\vspace*{+1cm}
 	
	\State \textbf{Begin}
	\For{$i = 1 \to numCrim$} 
		\State $crim\gets generateCriminal()$;
		\ForAll{timewindows tw}
		\State $topic \gets \Call{randomSelectTopic}{topics, biasPercentage}$; 
		\If{$i\leq numCrimTraining$}
			\State $numDocsTw \gets \Call{random}{0, numDocs/n, criminalPresentProb}$; 
		\Else
			\State $numDocsTw \gets \Call{random}{0, numDocs/n, criminalTestProb}$;
		\EndIf
			\For{$j = 1 \to numDocsTw$}
				\State $doc\gets$ \Call{generateDoc}{topic};
				\State \Call{associateDocument}{crim, doc};
			\EndFor
		\EndFor 
	\EndFor
	\State \textbf{End}
\end{algorithmic}
\end{algorithm}
\newpage
\normalsize
dove:
\begin{itemize}
	\item \emph{randomSelectTopic (topics, biasPercentage)} sceglie casualmente un argomento tra i 7 disponibili, ma con probabilità pari a \emph{biasPercentage} sceglie quello della finestra temporale precedente;
	\item \emph{random(0, numDocs/n, criminalPresentProb)} estrae un numero casuale tra 0 e \emph{numDocs/n} con probabilità \emph{criminalPresentProb};
	\item \emph{generateDoc(topic)} genera un documento relativo al topic fornito, seguendo l'algoritmo di seguito riportato.
\end{itemize}

\footnotesize
Siano:
\begin{algorithmic}
	\State $minNumWordsDoc\gets 1500$; \Comment{Numero minimo di parole rumore nel documento}
	\State $maxNumWordsDoc\gets 2000$; \Comment{Numero massimo di parole rumore nel documento}
			
	\State $topicWordMinOcc\gets 10$;	\Comment{Numero minimo di occorrenze per una parola chiave}
	\State $topicWordMaxOcc\gets 15$;	\Comment{Numero massimo di occorrenze per una parola chiave}
	\State $dictWordMinOcc\gets 1$;	\Comment{Numero minimo di occorrenze per una parola rumore}
	\State $dictWordMaxOcc\gets 3$;	\Comment{Numero massimo di occorrenze per una parola rumore}
	\State $topicWordsPercentage\gets 0.6$;	\Comment{\% di parole chiave inserite dell'argomento scelto}
	\State $listWordsTopic$ \Comment{Parole chiave associate all'argomento}
	\State $listWordsDict$ \Comment{Parole comuni del dizionario (rumore)}
	\State $numWordsTopic$ 
	{Numero di parole chiave associate all'argomento}
\end{algorithmic}\ 
\begin{algorithmic}
	\State \textbf{Begin}
	\State $doc\gets emptyDocument();$
	\State $numWords\gets random(minNumWordsDoc, maxNumWordsDoc);$
	\For{$i = 1 \to numWords$}
		\State $word\gets randomSelectWord(listWordsDict);$
		\State $removeWord(listWordsDict, word);$
		\State $numOcc\gets random(dictWordMinOcc, dictWordMaxOcc);$ 
		\For{$j = 1 \to numOcc$}
			\State $addWord(doc, word);$
		\EndFor
	\EndFor
	
	\State $numWords\gets topicWordsPercentage*numWordsTopic;$
	\For{$i = 1 \to numWords$}
		\State $word\gets randomSelectWord(listWordsTopic);$
		\State $removeWord(listWordsTopic, word);$
		\State $numOcc\gets random(topicWordMinOcc, topicWordMaxOcc);$ 
		\For{$j = 1 \to numOcc$}
			\State $addWord(doc, word);$
		\EndFor
	\EndFor
	
	\Return doc;
	
	 \textbf{End}
\end{algorithmic}
\normalsize
Il trattamento riservato alle parole chiave è dunque il medesimo di quelle rumore, fatta eccezione per la frequenza più elevata (tra 10 e 15 anziché tra 1 e 3).

Seguendo tale strategia, sono stati generati artificialmente 4 dataset cosi composti:
\begin{table}[H]
	\centering
	\footnotesize
	\begin{tabular}{|cccc|}
	\hline
	\textbf{\#} & \textbf{DB\_Name} & \textbf{\# Criminali di Training} & \textbf{\# Criminali di Test} \\ 
	\hline
	1 & DB\_Completo & 600 & 0 \\
	2 & DB\_Test1 & 590 & 10 \\
	3 & DB\_Test2 & 550 & 50 \\
	4 & DB\_Test3 & 500 & 100 \\
	\hline
	\end{tabular}
	\caption{Data Set su cui effettuare la sperimentazione.}
	\label{TAB_DB}
\end{table}
Per ogni criminale appartenente all'insieme di Test verranno eliminati i documenti le cui date rientrano nei periodi 1982-1985,1994-2001 (ossia nelle finestre con Id 4,7,8).
Una volta generati i dati, si è reso necessario individuare le modalità di esecuzione, in quanto il sistema prevede diverse fasi, ciascuna delle quali
eseguibile con diverse metodologie e con diversi parametri. Le combinazioni possibili risultano praticamente infinite, pertanto sono state fatte delle scelte al fine di ridurle ad un numero accettabile: 
\begin{itemize}
\item \textbf{Feature Selection}: selezione dei 30 termini con maggior varianza.
\item \textbf{Calcolo posizione semantica dei criminali}: MaxDensity con parametri SIGMA = 0.05, h (numero finestre passate da considerare) = 5;
\item \textbf{Clustering}: K-Means con stima del parametro k (numero di cluster da generare) = 7, pari cioè al numero di cluster realmente esistenti;
\item \textbf{Calcolo della posizione semantica dei cluster}: calcolo del centroide; 
\end{itemize}

Prima di passare alla definizione dei parametri per verificare la bontà dei risultati ottenuti, è necessario definire come applicare le tecniche precedentemente definite ai 4 dataset.
DB\_Completo rappresenterà il dataset di riferimento, utilizzato per verificare la correttezza dei risultati prodotti: su tutti i 600 criminali sono state applicate le tecniche di feature selection sull'insieme dei documenti generati, calcolate le posizioni semantiche dei criminali e da queste calcolati, per ogni finestra, i cluster.
In DB\_Test1, DB\_Test2, DB\_Test3 sono realizzate le stesse operazioni per i criminali di training ad esso associati (rispettivamente 590, 550, 500 criminali), mentre per ogni criminale da testare (rispettivamente 10, 50 100), verrà stimata la categoria criminale nelle finestre note e predetta in quelle sconosciute.

Per verificare se il cluster predetto dal sistema corrisponda a quello reale, ossia corrisponda al cluster in cui il criminale sarebbe inserito se il processo di clustering venisse realizzato su tutti i criminali (sia di Test che di Training), viene effettuato un confronto tra i cluster appartenenti al DB\_Completo e quelli appartenenti ai database contenenti i criminali da testare, associati alle stesse finestre temporali. Per ogni cluster appartenente al database di Test verrà dunque identificato il cluster più simile tra quelli del DB\_Completo, 
dove la similarità è calcolata attraverso la misura di Jaccard (vedi paragrafo \ref{predizione_fin_note}).  
 

Una volta fissati i parametri e gli algoritmi per calcolare le posizioni semantiche dei cluster necessari al processo di predizione, si definiscono di seguito i parametri che verranno fissati per realizzare questa seconda sperimentazione:
\begin{itemize}
	\item Numero di finestre temporali note entro cui andare a generare i cluster surrogati (\textit{n});
	\item Utilizzo delle soluzioni parziali per il calcolo delle finestre temporali ancora sconosciute \textbf{TwPrec}. TwPrec sarà true se nel processo di predizione verranno utilizzate soluzioni parziali, false altrimenti);
\end{itemize}

Una volta definite le modalità di esecuzione della sperimentazione, è importante scegliere delle misure in grado di fornire un'indicazione della qualità del risultato ottenuto. In particolare, si è deciso di considerare, come nella prima sperimentazione, due fattori:
\begin{itemize}
	\item tempo di esecuzione del processo;
	\item qualità delle previsione ottenute.
\end{itemize}

Per quanto riguarda l'analisi della qualità delle previsioni ottenute, verrà utilizzato il calcolo della precisione e richiamo su macro-media.
In particolare ad ogni cluster, appartenente alle finestre in cui è stato effettuato il processo di predizione, saranno associate quattro misure:
\begin{itemize}
\item \textbf{True Positive:} numero di criminali correttamente associati al cluster;
\item \textbf{False Positive:} numero di criminali erroneamente associati al cluster;
\item \textbf{True Negative:} numero di criminali correttamente non associati al cluster;
\item \textbf{False Negative:} numero di criminali erroneamente non associati al cluster;
\end{itemize}
Si definisce di seguito l'algoritmo utilizzato per il calcolo delle misure:
\footnotesize
\begin{algorithm}[H]
\caption{Calcolo per ogni cluster dei True Positive, False Negative e False Positive }
\label{calcoloTPFNFP}
\begin{algorithmic}
	\State $timeWindows\gets \Call{getTimeWindows}{ };$	\Comment{Lista delle time window}
	\State $listCriminals\gets \Call{getCriminalsTest}{ };$	\Comment{Numero di criminali di Test su cui effettuare il processo di predizione}
	\State $listClusters\gets \{\};$	\Comment{Insieme dei cluster appartenenti alle finestre coinvolte nella predizione}
	\vspace*{+1cm}
 	
	\State \textbf{Begin}
	\ForAll{ $criminal \in ListCriminals$}{
		 \ForAll{$timeWindow \in TimeWindows$}{
			\If{$criminal$ not in $TimeWindow$}{
				\State $listClusterCurrentTW\gets \Call{getClustersTW}{timeWindows}$ 
				\State $ \Call{InsertIntoListClusters}{listClusterCurrentTW}$
				\State $cluster\gets \Call{getClusterMoreProb}{criminal,listClusterCurrentTW}$
				\State $clusterCorrect\gets \Call{getCorrectCluster}{criminal,timeWindow}$
				\If{$cluster = clusterCorrect$}{
					\State $cluster.TP\gets cluster.TP +1$;
					\State $\Call{updateListClusters}{cluster}$
				\Else
					\State $cluster.FP\gets cluster.FP+1$
					\State $clusterCorrect.FN\gets clusterCorrect.FN+1$
					\State $\Call{updateListClusters}{cluster}$
					\State $\Call{updateListClusters}{clusterCorrect}$
				\EndIf}
				
			\EndIf}
		  \EndFor} 
	
 	\EndFor
 	}
 	
 	\State \Return $listClusters$;
\end{algorithmic}
\end{algorithm}
dove:
\begin{itemize}
\item $getClustersTW(timeWindows)$ restituisce i cluster appartenenti a quella time window e li inserisce in listClusters;
\item $InsertIntoListClusters(listClusterCurrentTW)$ inserisce in \textit{listClusterCurrentTW} i cluster non presenti in $listClusters$;
\item $getClusterMoreProb(criminal,listClusterCurrentTW)$ restituisce il cluster più simile a $criminal$ tra quelli in $listClusterCurrentTW$;
\item $getCorrectCluster(criminal,timeWindow)$ restituisce il cluster reale in cui $criminal$ è contenuto nella finestra $timeWindow$;
\item $updateListClusters(cluster)$ aggiorna $listCluter$ modificando gli attributi \textit{TP, FP, FN, TN} associati a $cluster$ 
\end{itemize} 

L'algoritmo \ref{calcoloTPFNFP} permette quindi di definire per ogni cluster $C$, appartenente ad una finestra sconosciuta, il valore delle misure True Positive ($TP_C$), False Positive ($FP_C$) e False Negative ($FN_C$), e calcolare la precisione e il richiamo associati ad esso:
\begin{equation}
\centering
Precision_C = \frac{TP_C}{TP_C + FP_C}	
\end{equation}
\begin{equation}
\centering
Recall_C = \frac{TP_C}{TP_C + FN_C}
\end{equation}

Le somme di tutte le precisioni e richiami, mediate per il numero di cluster analizzati permetteranno di definire la precisione e il richiamo dell'intero sistema.
\begin{equation}
\centering
Precision = \frac{\sum_{C \in listClusters}  Precision_C}{|listClusters|}	 
\end{equation}

\begin{equation}
\centering
Recall = \frac{\sum_{C \in listClusters}  Recall_C}{|listClusters|}	 
\end{equation}
dove \textit{listClusters} è la lista di cluster restituita dall'algoritmo \ref{calcoloTPFNFP}.

Ai fini della sperimentazione si è deciso di valutare l'algoritmo realizzato impostando \textit{n} a 2 o 3, con e senza l'utilizzo delle soluzioni parziali per il calcolo delle altre finestre ancora sconosciute, ottenendo così 16 possibili combinazioni.

Anche per questa sperimentazione si riportano le caratteristiche hardware e software utilizzate:
\begin{itemize}
	\item CPU Intel Core i5 2500K - Core: 4 - Frequenza: 4Ghz;
	\item RAM 8GB DDR3 1600Mhz;
	\item Hard Disk Western Digital 500GB 5400rpm;
	\item S.O. Microsoft Windows 7 64bit;
	\item DBMS PostgreSQL 9.2;
	\item Strawberry Perl (64-bit) 5.16.2.1;
	\item Java 1.7 update 9 - 64bit.
\end{itemize}


Si riportano di seguito i risultati ottenuti:

\textbf{Calcolo della precisione}

\begin{table}[H]
	\centering
	\footnotesize
	\begin{tabular}{|cccccp{0.1\textwidth}|}
	\hline
	\textbf{DataSet} & \textbf{TW 4} & \textbf{ TW 7} & \textbf{TW 8} & \textbf{average} & \textbf{time}\\
	\hline 
	DB\_Test1 & 0,81 & 0,81 & 1 & 0,87 & 0:09:20\\ 
	DB\_Test2 & 0,64 & 0,66 & 0,85 & 0,72 & 0:55:21\\ 
	DB\_Test3 & 0,82 & 0,72 & 0,87 & 0,80 & 1:50:11\\ 
	\hline 
	\end{tabular}
	\caption{Precisione calcolata sulle time windows sconosciute (TW 4, TW 7, TW 8) e tempo di esecuzione per n=2 e TwPrec=true}
\end{table}
\begin{table}[H]
	\centering
	\footnotesize
	\begin{tabular}{|cccccp{0.1\textwidth}|}
	\hline
	\textbf{DataSet} & \textbf{TW 4} & \textbf{ TW 7} & \textbf{TW 8} & \textbf{average} & \textbf{time}\\
	\hline 
	DB\_Test1 & 0,81 & 0,81 & 1 & 0,87 & 0:11:33\\ 
	DB\_Test2 & 0,64 & 0,66 & 0,85 & 0,72 & 0:52:31\\ 
	DB\_Test3 & 0,82 & 0,72 & 0,87 & 0,80 & 1:49:21\\ 
	\hline 
	\end{tabular}
	\caption{Precisione calcolata sulle time windows sconosciute (TW 4, TW 7, TW 8) e tempo di esecuzione per n=2 e TwPrec=false}
\end{table}

\begin{table}[H]
	\centering
	\footnotesize
	\begin{tabular}{|cccccp{0.1\textwidth}|}
	\hline
	\textbf{DataSet} & \textbf{TW 4} & \textbf{ TW 7} & \textbf{TW 8} & \textbf{average} & \textbf{time}\\
	\hline 
	DB\_Test1 & 0,74 & 0,81 & 1 & 0,85 & 0:11:28\\ 
	DB\_Test2 & 0,64 & 0,65 & 0,84 & 0,71 & 0:56:44\\ 
	DB\_Test3 & 0,77 & 0,71 & 0,87 & 0,78 & 1:50:00\\ 
	\hline 
	\end{tabular}
	\caption{Precisione calcolata sulle time windows sconosciute (TW 4, TW 7, TW 8) e tempo di esecuzione per n=3 e TwPrec=true}
\end{table}
\begin{table}[H]
	\centering
	\footnotesize
	\begin{tabular}{|cccccp{0.1\textwidth}|}
	\hline
	\textbf{DataSet} & \textbf{TW 4} & \textbf{ TW 7} & \textbf{TW 8} & \textbf{average} & \textbf{time}\\
	\hline 
	DB\_Test1 & 0,74 & 0,81 & 1 & 0,85 & 0:10:02\\ 
	DB\_Test2 & 0,6 & 0,66 & 0,85 & 0,70 & 0:52:22\\ 
	DB\_Test3 & 0,77 & 0,72 & 0,87 & 0,79 & 1:46:20\\ 
	\hline 
	\end{tabular}
	\caption{Precisione calcolata sulle time windows sconosciute (TW 4, TW 7, TW 8) e tempo di esecuzione per n=3 e TwPrec=false}
\end{table}

\begin{figure}
\centering
\subfigure{\includegraphics[scale=0.5]{Immagine31}}\qquad
\subfigure{\includegraphics[scale=0.5]{Immagine32}}
\subfigure{\includegraphics[scale=0.5]{Immagine33}}\qquad
\subfigure{\includegraphics[scale=0.5]{Immagine34}}
\caption{Confronto tra le precisioni calcolate sui data set TB\_Test1, TB\_Test2, TB\_Test3}
\end{figure}

\textbf{Calcolo del Richiamo}
\begin{table}[H]
	\centering
	\footnotesize
	\begin{tabular}{|cccccp{0.1\textwidth}|}
	\hline
	\textbf{DataSet} & \textbf{TW 4} & \textbf{ TW 7} & \textbf{TW 8} & \textbf{average} & \textbf{time}\\
	\hline 
	DB\_Test1 & 0,61 & 0,85 & 1 & 0,82 & 0:09:20\\ 
	DB\_Test2 & 0,52 & 0,53 & 0,82 & 0,62 & 0:55:21\\ 
	DB\_Test3 & 0,68 & 0,55 & 0,74 & 0,66 & 1:50:11\\ 
	\hline 
	\end{tabular}
	\caption{Richiamo calcolato sulle time windows sconosciute (TW 4, TW 7, TW 8) e tempo di esecuzione per n=2 e TwPrec=true}
\end{table}
\begin{table}[H]
	\centering
	\footnotesize
	\begin{tabular}{|cccccp{0.1\textwidth}|}
	\hline
	\textbf{DataSet} & \textbf{TW 4} & \textbf{ TW 7} & \textbf{TW 8} & \textbf{average} & \textbf{time}\\
	\hline 
	DB\_Test1 & 0,61 & 0,85 & 1 & 0,82 & 0:11:33\\ 
	DB\_Test2 & 0,52 & 0,53 & 0,82 & 0,62 & 0:52:31\\ 
	DB\_Test3 & 0,68 & 0,55 & 0,74 & 0,66 & 1:49:21\\  
	\hline 
	\end{tabular}
	\caption{Richiamo calcolato  sulle time windows sconosciute (TW 4, TW 7, TW 8) e tempo di esecuzione per n=2 e TwPrec=false}
\end{table}

\begin{table}[H]
	\centering
	\footnotesize
	\begin{tabular}{|cccccp{0.1\textwidth}|}
	\hline
	\textbf{DataSet} & \textbf{TW 4} & \textbf{ TW 7} & \textbf{TW 8} & \textbf{average} & \textbf{time}\\
	\hline 
	DB\_Test1 & 0,62 & 0,86 & 1 & 0,83 & 0:11:28\\ 
	DB\_Test2 & 0,48 & 0,5 & 0,82 & 0,60 & 0:56:44\\ 
	DB\_Test3 & 0,57 & 0,53 & 0,74 & 0,61 & 1:50:00\\  
	\hline 
	\end{tabular}
	\caption{Richiamo calcolato  sulle time windows sconosciute (TW 4, TW 7, TW 8) e tempo di esecuzione per n=3 e TwPrec=true}
\end{table}

\begin{table}[H]
	\centering
	\footnotesize
	\begin{tabular}{|cccccp{0.1\textwidth}|}
	\hline
	\textbf{DataSet} & \textbf{TW 4} & \textbf{ TW 7} & \textbf{TW 8} & \textbf{average} & \textbf{time}\\
	\hline 
	DB\_Test1 & 0,62 & 0,86 & 1 & 0,83 & 0:10:02\\ 
	DB\_Test2 & 0,48 & 0,53 & 0,82 & 0,61 & 0:52:22\\ 
	DB\_Test3 & 0,56 & 0,55 & 0,74 & 0,62 & 1:46:20\\ 
	\hline 
	\end{tabular}
	\caption{Richiamo calcolato  sulle time windows sconosciute (TW 4, TW 7, TW 8) e tempo di esecuzione per n=3 e TwPrec=false}
\end{table}

\begin{figure}[H]
\centering
\subfigure{\includegraphics[scale=0.5]{Immagine35}}\qquad
\subfigure{\includegraphics[scale=0.5]{Immagine36}}
\subfigure{\includegraphics[scale=0.5]{Immagine37}}\qquad
\subfigure{\includegraphics[scale=0.5]{Immagine38}}
\caption{Confronto tra i richiami calcolati sui data set TB\_Test1, TB\_Test2, TB\_Test3}
\end{figure}

Osservando i precedenti grafici e tabelle è possibile osservare come i valori medi della precisione e richiamo non variano sensibilmente al variare dei parametri n e TwPrec, oscillando tra il 78\% per n=3 e l'80\% per n=2 per la precisione e tra il 68\% e il 70\% per il richiamo, indipendentemente dal valore di TwPrec. Questo è probabilmente dovuto all'assenza di significative evoluzioni dei criminali nelle n finestre note utilizzate nel processo di evoluzione. Infatti i documenti associati ai criminali sono stati costruiti in modo che la probabilità che un criminale evolva da una categoria criminale (es. rapina) in un'altra (es. omicidio) è pari al 30\%. 





%\input{Capitoli/Ipsum}
%\appendix
%\input{Capitoli/Dolor}
% *****************************************************************
% Materiale finale
%******************************************************************
%% !TEX encoding = UTF-8
% !TEX TS-program = pdflatex
% !TEX root = ../Tesi.tex
% !TEX spellcheck = it-IT

%*******************************************************
% Introduzione
%*******************************************************
In questo lavoro di tesi è stata trattata la problematica della sicurezza, evidenziando l'importanza che essa riveste dal punto di vista sociale e le difficoltà sociali e tecniche incontrate nella realizzazione di alcuni progetti.
La trattazione effettuata, incentrata sull'estrazione di entità nominali da documenti testuali e sulla predizione ed evoluzione delle categorie criminali nel tempo, non pretende di essere esaustiva, ma piuttosto un punto di partenza per ulteriori sviluppi, sia teorici che sperimentali. 

Il sistema TB-CREDIS è stato esteso per permettere l'estrazione dei perpetratori di un crimine all'interno di documenti più o meno segretati, che possono spaziare da articoli giornalistici a report investigativi, e predire le evoluzioni dei criminali, relativamente alle loro attività, in quegli intervalli temporali per cui non si possiede alcun documento associato al soggetto in esame relativo a tal periodo. L'alta modularità del sistema ha permesso inoltre la facile integrazione della componente di interfaccia utente: l' utente può caricare un nuovo criminale nella collezione di dati, effettuare le operazioni di predizione e visualizzare graficamente i risultati prodotti e le evoluzioni del criminale sotto analisi nel tempo.

I risultati sperimentali prodotti si sono rivelati all'altezza delle aspettative e incentivano a proseguire gli studi in questa direzione in modo da individuare nuove tecniche che permettano di migliorare i risultati raggiunti in termini di tempi di elaborazione e di qualità. 
In particolare la componente di estrazione delle entità perpetratrici può essere estesa con nuove euristiche per permettere di estrarre le entità nominali da quei pattern ad oggi non considerati, o estrarre nuove entità quali nomi di possibili vittime, luoghi, proprietà private e tutto ciò che possa rivelarsi utile alle attività investigative. 
Per la componente di predizione possibili sviluppi futuri potrebbero riguardate l'analisi e individuazione dei documenti relativi allo stesso reato, commesso da un criminale, in modo da evitare che reati più ``documentati'' abbiano più importanza di altri reati commessi nel calcolo della posizione semantica del criminale stesso.
\bibliographystyle{plain}
\bibliography{bib}                % database di biblatex 

\end{document}