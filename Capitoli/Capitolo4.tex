% !TEX encoding = UTF-8
% !TEX TS-program = pdflatex
% !TEX root = ../tesi.tex
% !TEX spellcheck = it-IT

%************************************************

%************************************************


In questo capitolo si descriverà la progettazione e l'esecuzione della sperimentazione. Si partirà pertanto dai dati su cui quest'ultima è stata effettuata, proseguendo con la scelta delle modalità di esecuzione più interessanti e concludendo con una serie di tabelle e grafici contenenti i risultati ottenuti, opportunamente commentati.  
\section{Sperimentazione modulo estrazione dei perpetratori}
L'obiettivo di questa prima sperimentazione è valutare l'efficacia del sistema analizzato al fine di comprendere le cause di un eventuale successo o insuccesso delle tecniche utilizzate e quindi decidere se proseguire gli studi in questa direzione o definire strategie alternative.


Nonostante in grande interesse in quest'area di ricerca, vi sono ad oggi, pochi corpora disponibili per valutare un sistema di scoperta di eventi da Twitter
La sperimentazione del sistema è stata effettuata sfruttando i dati appartenenti al database GTD (Global Terrorism Database)\cite{GTD}.
Questo database open-source, reso disponibile dal National Consortium for the Study of Terrorism and Responses to Terrorism (START), contiene informazioni su eventi terroristici avvenuti nel mondo nel periodo 1979-2011. A differenza di altri database, GTD contiene dati su incidenti terroristici nazionali e internazionali, per un totale di 104.000 istanze. Per ogni incidente sono disponibili una serie di informazioni quali data, luogo dell'incidente, armi utilizzate, numero di vittime e nomi degli individui o gruppi responsabili.
Si descrivono di seguito le caratteristiche principali del database GTD:
 


