\chapter{dbpedia}
Wikipedia \footnote{https://en.wikipedia.org/} è divenuta una delle maggiori risorse di conoscenza disponibili nel web, ed è manutenuta da migliaia di utenti (collaboratori). Gli articoli Wikipedia sebbene composti prevalentemente da testo, contengono informazioni semi-strutturate come: template infobox , informazioni sulla categorizzazione dell'articolo, immagini, geo-coordinate e link sia verso altre pagine web sia verso altre pagine wikipedia.
Gli infobox sono tabelle di coppie attributo valore, che mostrano i dati più rilevanti di ciascuna pagina wikipedia. Il progetto DBpedia \cite{Bizer:2009:DCP:1640541.1640848} estrae dati strutturati da wikipedia tramite un extraction framweork open source e li unisce in una base di conoscenza multi dominio e multi lingua. Per ogni pagina presente in wikipedia, viene associato un \emph{Uniform Resource Identifier (URI)} in DBpedia per identificare un'entità o un concetto descritto dalla corrispondente pagina Wikipedia della versione inglese. Durante il processo di trasformazione, i dati semi-strutturati come i campi infobox,categorie, pagelinks sono convertiti in triple RDF e aggiunte alla base di conoscenza come proprietà dell'entità identificata dall URI.  Per rendere omogenea la descrizione delle informazioni, è stata sviluppata un ontologia e sono state definite le corrispondenze fra le proprietà presenti negli infobox e l'ontologia.
L'ontologia DBpedia consisite di 320 classi e descritte da 1650 proprietà. Le classi organizzate mediante una gerarchia sussuntiva dove \emph{owl:Thing} è la classe più generale. Poiché Il sistema di Wikipedia infobox si è evoluto in maniera decentrata, talvolta accade ad esempio che si usino diversi template per la stessa tipologia di entità (class) o si usino nomi diversi per descrive lo stesso attributo (es placeOfBirth o birthPlace). 
 L'allineamento tra i template infobox e l'ontologia a causa di queste eterogeneità presenti nella nomenclatura, non è quindi completamente automatico, ma si basa anche su mapping definiti manualmente, forniti dalla comunità di DBpedia. Ad esempio ‘date of birth’ and ‘birth
date’ sono entrambi mappati con la proprietà birthDate. 
\subsection{Accedere a DBpedia}
La base di conoscenza DBpedia è disponibile sul web sotto GNU Free Documentation, e può essere consultata mediante varie modalità :
\begin{itemize}
\item \emph{Linked Data: linked data è la metodologia di pubblicazione  dei dati RDF nel web, che utilizza gli URI http come identificativo delle risorse e il protocollo HTTP per ritrovare al descrizione rdf delle risorse. Quando si accedere ad un URI di una risorsa DBPedia mediante un semantic web agent si ottiene la descrizione rdf della risorsa mentre se si utilizza un semplice web-browser si otterrà una vista html descrizione.
} 
\item \emph{Sparql Endpoint } \'E fornito un endopoint mediante il quale si può interrogare la base di conoscenza tramite il protocollo SPARQL.
\item \emph{RDF dumps} la base di conoscenza è stata suddivisa in varie parti in base agli rdf-predicate 
\end{itemize}

 